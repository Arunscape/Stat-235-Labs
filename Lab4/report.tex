%preamble
\documentclass[letterpaper]{article}
\synctex=1
\usepackage{graphicx}
\graphicspath{ {images/} }

\usepackage{lipsum}
\usepackage{float}
% \bibliographystyle{IEEEtran}
% \bibliographystyle{ieeetr}

\usepackage{amssymb}

\usepackage{siunitx}

\usepackage{multirow}
% for merging table cells I think

\usepackage{tabularx}
% allows for linewrap within cells
\newcolumntype{Y}{>{\centering\arraybackslash}X}

\usepackage{fancyhdr} %header
\fancyhf{}
\fancyhead[R]{Arun Woosaree XXXXXXX}
\renewcommand\headrulewidth{0pt}
\fancyfoot[C]{\thepage}
\renewcommand\footrulewidth{0pt}
\pagestyle{fancy}

% make subsection use letters
\renewcommand{\thesubsection}{\thesection.\alph{subsection}}


\usepackage{amsthm}
\newtheorem*{clt}{Central Limit Theorem}

%actual document
\begin{document}

% \maketitle %insert titlepage here
\begin{titlepage}
 \begin{center}
  \vspace*{1cm}
  \Huge
  Stat 235
  \vspace{1cm}
  
  Lab 4
  \vspace{1cm}
  
  WOOSAREE, Arun
  \vspace{1cm}
  
  \Huge
  Lab EL12
  \vspace{1cm}
  
  TA: Jessa Marley
  \vspace{1cm}
  
  \today
  \vfill
 \end{center}
\end{titlepage}

\section{}%1

\subsection{}%1a
Keeping other parameters constant, changing the confidence level yields the folowing:
\begin{table}[H]
 \centering
 \begin{tabular}{|c|c|}
  \hline
  Confidence Level & Margin of Error \\ \hline
  0.90             & 0.300308        \\ \hline
  0.95             & 0.357839        \\ \hline
  0.99             & 0.470280        \\ \hline
 \end{tabular}
 \caption{My caption}
 \label{1a}
\end{table}
How does the margin of
error change as the confidence interval increases? Explain briefly.
As seen in Table \ref{1a} above, the Margin of Error increases as the
Confidence Level is increased. This makes sense because............


\subsection{}%1b

\begin{table}[H]
 \centering
 \begin{tabularx}{\textwidth}{|c|Y|}
  \hline
  Confidence Level & Observed Fraction of Intervals That Failed to Cover the Hypothesized Population Mean \\ \hline
  0.90             & 0.11                                                                                 \\ \hline
  0.95             & 0.06                                                                                 \\ \hline
  0.99             & 0.02                                                                                 \\ \hline
 \end{tabularx}
 \caption{My caption}
 \label{1b}
\end{table}

Are the observed counts consistent with the values predicted
by the theory? Explain briefly.
looks like you got some learnin to do....

\section{}%2
$$ H_0: \mu=64 \quad vs. \quad H_A: \mu \neq 64 $$

\subsection{}%2a

\begin{table}[H]
 \centering
 \begin{tabularx}{\textwidth}{|Y|Y|Y|}
  \hline
  Level of Significance & Number of Samples That Led to the Rejection of $H_0$ & Observed Fraction of Samples \\ \hline
  0.10                  & XXXX                                                 & XXXX                         \\ \hline
  0.05                  & XXXX                                                 & XXXX                         \\ \hline
  0.01                  & XXXX                                                 & XXXX                         \\ \hline
 \end{tabularx}
 \caption{My caption}
 \label{2a}
\end{table}

How does the number of samples change as
the level of significance increases? Explain briefly.

\subsection{}%2b

Write your null hypothesis. (SHould have a solid understanding of p-values for this)

Compare the outcome of the test at the 5\% level of significance with the 95\% confidence intervals
that failed to cover the mean of 64 for each sample. Repeat the exercise with the 1\% level of
significance and the 99\% confidence intervals. What do you conclude about the relationship
between confidence intervals and two-sided tests?

\section{}%3

\subsection{}%3a

\begin{table}[H]
 \centering
 \begin{tabular}{|c|c|}
  \hline
  \multicolumn{2}{|c|}{Alloy 1}          \\ \hline
  Mean                     & 65.09       \\ \hline
  Standard Error           & 0.360980466 \\ \hline
  Median                   & 64.6        \\ \hline
  Mode                     & 63.8        \\ \hline
  Standard Deviation       & 1.977171438 \\ \hline
  Sample Variance          & 3.909206897 \\ \hline
  Kurtosis                 & 0.042639157 \\ \hline
  Skewness                 & 0.718164135 \\ \hline
  Range                    & 8.2         \\ \hline
  Minimum                  & 61.7        \\ \hline
  Maximum                  & 69.9        \\ \hline
  Sum                      & 1952.7      \\ \hline
  Count                    & 30          \\ \hline
  Confidence Level(95.0\%) & 0.738287948 \\ \hline
 \end{tabular}
 \caption{My caption}
 \label{3a1}
\end{table}

\begin{table}[H]
 \centering
 \begin{tabular}{|c|c|}
  \hline  \multicolumn{2}{|c|}{Alloy 2}  \\ \hline
  Mean                     & 65.27333333 \\ \hline
  Standard Error           & 0.167601973 \\ \hline
  Median                   & 65          \\ \hline
  Mode                     & 64.9        \\ \hline
  Standard Deviation       & 0.917993815 \\ \hline
  Sample Variance          & 0.842712644 \\ \hline
  Kurtosis                 & 9.565960304 \\ \hline
  Skewness                 & 2.914366915 \\ \hline
  Range                    & 4.5         \\ \hline
  Minimum                  & 64.5        \\ \hline
  Maximum                  & 69          \\ \hline
  Sum                      & 1958.2      \\ \hline
  Count                    & 30          \\ \hline
  Confidence Level(95.0\%) & 0.342784524 \\ \hline
 \end{tabular}
 \caption{My caption}
 \label{3a2}
\end{table}

\begin{table}[H]
 \centering
 \begin{tabular}{|c|c|}
  \hline
  \multicolumn{2}{|c|}{Alloy 2 + Treatment} \\ \hline
  Mean                     & 66.82333333    \\ \hline
  Standard Error           & 0.108350573    \\ \hline
  Median                   & 66.75          \\ \hline
  Mode                     & 66.9           \\ \hline
  Standard Deviation       & 0.593460531    \\ \hline
  Sample Variance          & 0.352195402    \\ \hline
  Kurtosis                 & 7.169408133    \\ \hline
  Skewness                 & 2.210176389    \\ \hline
  Range                    & 3.1            \\ \hline
  Minimum                  & 66             \\ \hline
  Maximum                  & 69.1           \\ \hline
  Sum                      & 2004.7         \\ \hline
  Count                    & 30             \\ \hline
  Confidence Level(95.0\%) & 0.221601804    \\ \hline
 \end{tabular}
 \caption{My caption}
 \label{3a3}
\end{table}

Paste the summary statistics into your report and
rep
ort the 95\% confidence interval
for each alloy. Use the summaries to compare the two alloys.
Which of the two alloys has
better strength qualities? Explain briefly.

\subsection{}%3b
According to the specifications, the mean strength of each alloy is required to exceed 64 ksi. Is
there any indication that the mean strength of either alloy is below the required threshold value of
64 ksi? Refe
r to the 95\% confidence interval for each alloy to answer the question. Explain briefly.

\section{}%4
Do the data provide evidence that the mean strength of each alloy exceeds the
threshold value of 64 ksi?  Now you will answer the question by carrying out the
appr opriate statistical tests.
\subsection{}%4a
Carry out an appropriate test to check the above claim using the data for each
alloy. In particular, state  the null and alternative hypotheses in terms of the
population parameters, obtain the value of the test  statistic , specify the
distribution of the test statistic under the null hypothesis, and obtain the  p
- value of  the test. What do you conclude? Notice that as there is no
appropriate feature in  Data Analysis to  carry out the test directly,  so  you
will have to calc ulate the value of the test statistic and the  corresponding
p - value by entering appropriate formulas into Excel worksheet.  Lab 3
Instructions may  be useful in this part.

\subsection{}%4b
What are the assumptions about the distribution of strength required to make t
he tests in part (a) valid?  Do the assumptions hold? Explain briefly. It is not
required to verify the assumptions with Excel.

\section{}%5
In this part , you will compare the mean strength of ALLOY 1 and ALLOY 2 rods.
Do the data provide  any evidence of a diffe rence in the mean strengths of
ALLOY 1 and ALLOY 2 rods?

\subsection{}%5a
Answer the above question by carrying out the appropriate test in the Data
Analysis menu. Before  you choose an appropriate test, you might refer to the
output in Question  3 to decide  what test would be appropriate.  In particular,
state the null and alternative hypotheses in terms of the  population
parameters, obtain the value of the test statistic, specify the distribution of
the test statistic  under the null hypothesis, and obtain the  p - value o f the
test. What do you conclude?

\subsection{}%5b
What are the assumptions about the distribution of strength required to make the
tests in part (a) valid?  Do the assumptions hold?  Explain briefly. It is not
required to verify the assumptions with Excel.

\section{}%6
Th e  thirty  ALLOY  2  rods  were  subjected  to  a  combination  of  high
pressure  and  temperature.  In  this  question , you will estimate the effect
of the treatment on the mean strength of the rods.

\subsection{}%6a
Do the data provide evidence that the treatment increased the mean s trength of
the ALLOY 2  rods? Answer the question by carrying out an appropriate test in
Excel.  In particular, state  the null  and alternative hypotheses in terms of
the population parameters, obtain the value of the test  statistic,  specify the
distributio n of the test statistic under the null hypothesis, and obtain the  p
- value of the test.  What do you conclude?

\subsection{}%6b
Use the  Descriptive Statistics feature in  the  Data Analysis menu to obtain a
95\%  two - sided  confidence interval for the mean change in strength o f ALLOY
2 rods after the treatment.   First create a new variable , EFFECT , defined as
the difference in strength between ALLOY 2 + TREATMENT rods and ALLOY 2 rods. Is
the interval consistent with the  test  outcome in part  (a)? Explain briefly.

\subsection{}%6c
What are the  assumptions necessary to make the test in part (a) and confidence
interval in part (b)  valid?  Do the assumptions hold?  Explain briefly.

\subsection{}%6d
Is the effect of the treatment independent of the initial strength of the rods?
In order to answer the  question, obtain the plot of the variable EFFECT versus
ALLOY 2 measurements. What do you  conclude?

% \begin{figure}[H]
%  \centering
%  \includegraphics[width=\textwidth]{q5.png}
%  \caption{Average number of flaws in samples of 30 panels from 1800 plastic panels.}
%  \label{5a}
% \end{figure}

\end{document}
