%preamble
\documentclass[letterpaper]{article}
\synctex=1
\usepackage{graphicx}
\graphicspath{ {images/} }

\usepackage{lipsum}
\usepackage{float}
% \bibliographystyle{IEEEtran}
% \bibliographystyle{ieeetr}

\usepackage{amssymb}

\usepackage{siunitx}

\usepackage{multirow}
% for merging table cells I think

\usepackage{fancyhdr} %header
\fancyhf{}
\fancyhead[R]{Arun Woosaree XXXXXXX}
\renewcommand\headrulewidth{0pt}
\fancyfoot[C]{\thepage}
\renewcommand\footrulewidth{0pt}
\pagestyle{fancy}

% make subsection use letters
\renewcommand{\thesubsection}{\thesection.\alph{subsection}}

%actual document
\begin{document}

% \maketitle %insert titlepage here
\begin{titlepage}
 \begin{center}
  \vspace*{1cm}
  \Huge
  Stat 235
  \vspace{1cm}
  
  Lab 2
  \vspace{1cm}
  
  WOOSAREE, Arun
  \vspace{1cm}
  
  \Huge
  Lab EL12
  \vspace{1cm}
  
  TA: Jessa Marley
  \vspace{1cm}
  
  \today
  \vfill
 \end{center}
\end{titlepage}

\section{Normal Density}

\subsection{}%1a
% Use normal density sheet. Manipulate the graph and describe the changes.
% Describe change as sigma increases in terms of process
% Effect of the change on alloy strength?



As $\sigma$ increases, there is more variation in the tensile strength (makes
sense, since $\sigma^2=V(x)$). This is seen visually as the curve flattening as
a result of the frequency count of the mode being lowered, and the frequency
counts at the more extreme ends being increased. In terms of tensile strength,
an increase in $\sigma$ does not change the mean but it increases the variation
in tensile strength. This does, however mean that less of the alloy produced
will be around the mean tensile strength, and it would increase the frequency of
alloys exceeding the tolerances.

\subsection{}%1b
% Use normal density sheet. Manipulate the graph and describe the changes.
% Describe change as mu increases
% Effect of the change on the fraction of unacceptable alloys?

the mean changes

As $\mu$ increases, the fraction below 275 \textbf{increases/decreases}
and the fraction below 295 \textbf{increases/decreases}.

% \begin{figure}[H]
%  \centering
%  \includegraphics[width=\textwidth]{x.png}
%  \caption{}
%  \label{}
% \end{figure}

\section{How changes in the Mean and Std. Deviation affect the fraction of alloy slabs that do not meet the TS Specifications.} %2
% #2
% use the normal probablities sheet

\begin{table}[H]
 \centering
 \begin{tabular}{c|c|c|c|}
                      & Parameters                                & Problem                                            & Answer \\ \hline
  \multirow{2}{*}{a)} & $\mu=285$ and $\sigma=5$                  & Fraction of unacceptable                           & 0      \\ \cline{2-4}
                      & $\mu=283$ and $\sigma=5$                  & Fraction of unacceptable                           & 0      \\ \hline
  b)                  & $\mu=285$ and $\sigma=6$                  & Fraction of unacceptable                           & 0      \\ \hline
  \multirow{2}{*}{c)} & \multirow{2}{*}{$\mu=285$ and $\sigma=5$} & Within 1 std. deviation                            & 0      \\ \cline{3-4}
                      &                                           & Within 2 std. deviations                           & 0      \\ \hline
  \multirow{2}{*}{d)} & \multirow{2}{*}{$\mu=285$ and $\sigma=5$} & Strength exceeded by 95\%                          & 0      \\ \cline{3-4}
                      &                                           & Strength exceeded by 99\%                          & 0      \\ \hline
  e)                  & $\mu=285$                                 & $\sigma$ so that 1\% have $TS < 275$ or $TS > 295$ & 0      \\ \hline
 \end{tabular}
 \caption{My caption}
 \label{my-label}
\end{table}

\section{Random Number Generator} %3

\subsection{} %generate the random numbers
% number of unacceptable slabs?
% consistency with #2a?

\subsection{}
%table
% what is the 68-95--99.7 rule?
% are the generated random numbers consitent with this rule?

\begin{table}[H]
 \centering
 \begin{tabular}{|c|c|c|c|}
  \hline
  \textbf{k} & \textbf{Within k Std. Deviations of the mean $\mu=285$} & \textbf{Frequency} & \textbf{Relative Frequency} \\ \hline
  1          & (0,0)                                                   & 0                  & 0                           \\ \hline
  2          & (0,0)                                                   & 0                  & 0                           \\ \hline
  3          & (0,0)                                                   & 0                  & 0                           \\ \hline
 \end{tabular}
 \caption{My caption}
 \label{my-label}
\end{table}

\subsection{} % distribution of the standardized values?
It's a normalized standard distibution.

\subsection{} % use data analysis -> descriptive statistics tool
% mean and std deviation fo the standardized values?
% consistency with normal distribution theory?
Using the \textit{Descriptive Statistics Tool, we find that}
$${mean} = XX.XXXX$$
$${std.\ deviation} = XX.XXXX$$

\section{Changes in Manufacturing Process}%4

\subsection{Summary Statistics} % use data analysis -> descriptive statistics tool
%4a
% consitency with target parameters?
Using the \textit{Desctiptive Statistics Tool, we find that}
$${mean} = XX.XXXX$$
$${std.\ deviation} = XX.XXXX$$
This is consistent / inconsistent

\subsection{Histogram}
%4b
% does it follow a normal distribution?
% does it support the assumption of having a normal distribution

\subsection{} %use the normal probablities sheet
%4c
% can the manufacturer reach the goal of less than 5% unacceptable?

\subsection{} % summarize your findings on TS. Are the changes acceptable?
%4d

\section{Binomial Probabilities}
%5

\subsection{} % use the binomial probablities sheet.
%5a
% parameters of the distibution?
% binomial probability?

\subsection{} % use the binomial probablities sheet.
%5b
% parameters of the binomial distribution
% (the value of p calculated with the normal probablities sheeet)
% binomial probability?














\end{document}
